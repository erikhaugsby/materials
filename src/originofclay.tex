\chapter{Origin of Clay}
Clay is a product of the continuous weathering of the Earth's surface. Our 
study of clay begins with how the planet Earth was formed.

The Earth was created some 5,000 million years ago, according to scientific 
theory. At first it was a gaseous molten mass, which slowly contracted. While 
the mass was still molten, the heavier materials like iron and nickel sunk to 
the centre of the Earth. As the hot Earth gradually cooled, a layer of solid 
materials formed the crust.The crust feels very secure to us, but it is only 
about 40 km thick, floating on a 3000 km deep layer of molten materials.

Slow currents in this molten sea cause whole continents to move at speeds of up 
to 2 cm per year. Where continents grind against each other, earthquakes and 
volcanoes occur. The map shows how all the big continent shave 
changed location during the last 200 million years. Africa was then so far to 
the South that it was covered by polar ice masses. The Indian plate moved 
North, and when it collided with the Asian continent the highest mountains on 
Earth- the Himalayas - were formed.In this way the crust of the Earth has 
changed continuously since it was formed. Where we see mountains today there 
may formerly have been wide oceans, and tropical forests may have been covered 
with arctic ice million of years ago. weathering: Weathering also causes major 
changes. Solid rocks are broken up by the alternate action of sun, rain and 
ice. The resulting small rock particles rocks are carried away by water, and 
even mountains wear away in a few million years. Clay and many other ceramic 
raw materials are produced by this process.
%-------------------------------------------------------------------------------
\section{Minerals and Rocks}
\subsection{Minerals}
A mineral is a substance which has a uniform chemical composition, in the form 
of one or many crystals. Quartz and feldspar are minerals, and so is salt. Salt 
crystals (sodium chloride) have a cubic shape which can easily be seen under a 
magnifying glass. When salt is dissolved in a glass of water and left for a 
while crystals will form slowly as the water evaporates.
%-------------------------------------------------------------------------------
\subsection{Rocks}
Most rocks are made up of several different minerals, though some rocks like 
gypsum only consist of one mineral. The rock named granite contains the 
minerals quartz, feldspar and mica, and the individual crystals can be seen 
clearly with a magnifying glass. Rocks can be arranged in three major groups; 
igneous rocks, sedimentary rocks and metamorphic rocks.
%-------------------------------------------------------------------------------
\subsubsection{Igneous Rocks}
When the young Earth slowly started to cool, different minerals formed crystals 
in the mass of molten rocks (also named magma). A variety of crystalline rocks 
were formed, according to the different conditions of their locality. Thus the 
igneous rock called basalt was created at a great depth, and contains little 
feldspar compared to granite, which formed near the surface.

If the rock cooled very slowly, the crystals had time to grow large, whereas 
rapid cooling produces small crystals. This process is still going on today 
where movement in the crust of the Earth causes deep layers of molten materials 
to rise to the surface. An erupting volcano lets out hot magma on the surface, 
where it cools quickly. The resulting volcanic rocks have microscopic size 
crystals, since the rapid cooling allows little time for the crystals to grow.
%-------------------------------------------------------------------------------
\subsubsection{Sedimentary Rocks}
Sedimentary rocks are made of materials produced by the crumbling of old rocks. 
All rocks eventually break up in the course of time when exposed to weather, 
and the broken up rock particles are carried away by water. These particles of 
clay and sand are transported to lower lying areas, or to the sea where they 
settle one layer upon the other. In the span of millions of years, the growing 
weight of sediments causes the deeper layers to compact and gradually turn into 
rocks, called sedimentary rocks. Much later, the movement of landmasses 
sometimes turns the whole area upside down, so that the old sea floor, with its 
sedimentary rocks, becomes a new range of mountains.

The upper part of new mountains consists of sedimentary rocks resting on deeply 
set igneous rocks. After some millions of years, the upper sedimentary rocks 
erodes away by weathering and the deeper igneous rocks are exposed.

Sedimentary rocks like sandstone and shale can often be recognized by their 
layered structure. Limestone is a sedimentary rock created by the left over 
skeletons of billions of small animals that lived in the ancient seas. Gypsum 
is formed by chemical sedimentation, in areas where seawater evaporated on a 
large scale. This produced a high concentration of gypsum which formed crystals 
in a fashion similar to the formation of salt crystals in a glass of salty 
water.
%-------------------------------------------------------------------------------
\subsubsection{Metamorphic Rocks}
Igneous and sedimentary rocks are sometimes changed into new forms by high 
temperatures and high pressure. Marble is an example of a metamorphic rock that 
is formed from a sedimentary rock named limestone.
%-------------------------------------------------------------------------------
\subsection{Rock Cycle}
Fig.~\ref{fig:} shows how one continental plate moves under another, causing 
some rocks to be exposed to high temperatures and pressure. Sedimentary or 
metamorphic rocks melt, and may later return to the surface as igneous rocks. 
These later will erode to become clay and sedimentary rocks. Fig.~\ref{fig:} 
shows this rock cycle graphically.
%-------------------------------------------------------------------------------
\section{Formation of Clay}
The formation of clay from rock is a most common event, taking place daily 
everywhere in the world. If a piece of granite is picked up and broken in two, 
the fresh faces of the stone will show a shiny surface and the crystals of the 
different minerals can be identified. The black shaded crystals are mica. The 
yellow, white or red colored crystals with a pearly shine are different types 
of feldspar. The clear colorless crystals are quartz. The weathered surface of 
the granite will most probably show a rough surface with many holes, where the 
soluble feldspar crystals have been washed away by rain, whereas the less 
soluble crystals of mica and quartz remain. This is the beginning of the 
process of changing feldspar into clay.

Weathering breaks up the granite rocks and enables the water to wash away the 
soluble soda, potash or lime parts of the feldspar. These soluble parts are 
carried away by water and the soda ends up adding to the salt of the oceans. 
The process is shown in fig.~\ref{fig:}. Most of the remaining alumina and 
silica of the feldspar combines with water and forms a new mineral: clay. Some 
of the silica from the feldspar does not take part in formation of clay and 
forms instead another mineral known as quartz.
%-------------------------------------------------------------------------------
\section{Primary Clay}
Clays which have not moved from the location of their parent rock are known as 
primary clays. Kaolin (also called China clay) is a primary clay.
%-------------------------------------------------------------------------------
\subsection{Steam and Acid}
In some cases the parent rock was exposed to steam from volcanic activity or to 
acid seeping down from above. The acid or hot steam slowly changed the rock 
into clay, quartz and mica in the same location. Such a deposit may only 
contain 25\% clay, and in some cases where the granite rock is only partly 
changed the clay content may be 10\% or less.
%-------------------------------------------------------------------------------
\subsection{Kaolin}
Kaolin clays are pure, without impurities like iron oxide and limestone. 
Therefore, they have a high melting point (about 1780\degree C) and they fire 
to a white colour. Kaolin has little plasticity due to its large particle size.
%-------------------------------------------------------------------------------
\section{Secondary Clay}
Clays which have been removed from their place of origin and have settled 
somewhere else are called secondary clays. Fig.~\ref{fig:} illustrates this 
process; rain washes the clay out from the site of its parent rock and the clay 
is carried downhill by rivers and streams.
%-------------------------------------------------------------------------------
\subsection{Grading}
A rapid river can even carry small stones, but as the flow of the river slows 
down, the heavy particles will start to settle, then the coarser sand, and 
finally the fine sand. The clay remains suspended in the water and will only 
settle where the river flows into a lake or a sea. At the mouth of the river, 
the silt and coarser clay will settle, whereas the fine clay will only settle 
further away. In this way, the original materials of the erosion process are 
graded according to their particle size (fig.~\ref{fig:}).
%-------------------------------------------------------------------------------
\subsection{Grinding}
The clay may travel thousands of kilometers before it settles. During transport 
the clay particles are subjected to the grinding action of pebbles in the 
streams. This works very much like grinding in a ball mill and makes the clay 
particles still smaller, thereby adding plasticity to the final clay.
%-------------------------------------------------------------------------------
\subsection{Impurities}
The river or stream carrying the clay will also pick up all kinds of impurities 
on its way through the landscape. These impurities will later be deposited 
together with the clay. It is very rare to find a pure secondary clay, i.e. a 
white firing plastic clay.

The most common impurity is iron oxide (rust) which gives the fired clay a 
buff, red or brown colour. Iron oxide exists in slightly different forms which 
may be yellow, red, grey or brown in the raw clay. However, the different iron 
oxides all turn into red iron oxide (\ce{FE2O3}) during an oxidizing firing. 
The clay may also contain organic matter which was added in the form of leaves 
and other vegetation from contemporary forests. This organic matter (carbon) 
gives the raw clay a grey or black colour, but as the carbon normally burns 
away during firing it does not affect the fired colour of the clay. Black clay 
is usually not black after firing. Other impurities like feldspar, mica and 
limestone lower the melting point of clay.

In general, secondary clays are plastic and have a lower melting point compared 
to primary clays, due to their impurities. The quality of secondary clays 
varies from one deposit to another, and even within the same deposit the 
quality of a clay can vary considerably.
%-------------------------------------------------------------------------------