\chapter{Prospecting and Mining Clay}
\section{Prospecting Clay}
Pottery clay can be found in most countries. In areas where pottery or 
brick-making have a long history suitable clay deposits will already be well 
known. However, introduction of new techniques like glazing or high temperature 
firing may call for new types of clay. In some countries the land may still be 
virgin from a potter's point of view and before any production facilities are 
established a reliable source of good pottery clay must be found.
%-------------------------------------------------------------------------------
\subsection{Practical People}
It is worth talking to people who make water wells, and builders of dams and 
roads. They should have first-hand information about the soil of the region. 
Farmers in the area will know about the upper layers of soil on their fields. 
Sometimes clay is used for other purposes, like whitewashing houses or 
medicine. In Tanzania iron smelters use a highly refractory clay for their 
furnaces, and in Nepal the brass makers use a local fireclay for their casting 
moulds.
%-------------------------------------------------------------------------------
\subsection{Prospecting}
When setting out to explore the countryside one should bear in mind how nature 
created clay. Recent deposits of plastic clay are most likely found in the 
plains and valleys, or along rivers. They are often close to the surface. Older 
secondary deposits may be found in the hills where the land was raised and 
folded, millions of years after the clay was deposited. Such deposits may be 
covered with a thick layer of other materials.
%-------------------------------------------------------------------------------
\subsubsection{River Banks}
A good potter's clay will often be covered by several meters of overburden. 
Instead of digging test holes through the overburden at random, a general idea 
of the deeper soils in the area can be obtained by examining the soils exposed 
in river banks, escarpments and cut areas where a road or a railway was made 
through a hill. Quarries, wells and ditches should be examined as well. 
Termites bring soil up from below, and the material from termite hills 
indicates the quality of soil located 1--2 meters down.
%-------------------------------------------------------------------------------
\subsubsection{Field Testing}
In the field a few simple tests can establish whether the clay is worth 
examining further. First, take a sample from about 30 cm inside the exposed 
surface, and mix it with water. If it turns into a plastic sticky mass it is 
clay. Then knead it well and form a rope the thickness of a pencil. If you can 
bend this rope around two fingers without seeing big cracks, the clay has 
plasticity (fig~\ref{fig}).

If you bite the plastic clay gently with your front teeth you will get an idea 
of how finely grained the sand content is.

Then rub a sample of dry clay in the palm of your hand until the fine particles 
are rubbed away. What remains is the grit content, which may be particles of 
feldspar, quartz or mica.

Small pieces of limestone will cause trouble in pot-making, so the deposit and 
its vicinity should be examined for signs of white or grey limestone. Lime 
powder thoroughly mixed with the clay will lower its melting point and is often 
used for low-temperature pottery. But a piece of limestone the size of a 
pinhead will crack the pots or bricks when they are exposed to moisture after 
firing. Screening the clay through a fine sieve reduces this problem, but if 
possible, it is better to look for another clay. Lime content can be tested by 
putting a few drops of dilute hydrochloric acid on a sample. If there is lime, 
bubbles will form from the reaction.
%-------------------------------------------------------------------------------
\subsubsection{Sample Collection}
If the clay, after field testing, has proven to be of interest samples should 
be collected for further and more thorough testing. The quality of clay from 
different places in the same deposit will differ slightly, so in order to get a 
representative sample, clay from 4 different spots within a few meters' 
distance are dug out. The clay should not be taken from the exposed surface of 
the deposit but rather from 30--50 cm inside since the clay at the surface may 
be contaminated with other soils or washed out by rain.

The 4 samples are mixed well at the location and a sample of about 5 kg is then 
packed and labeled with its location .

Make a sketch of the location as accurately as possible, indicating features of 
the landscape like big trees or rocks. If a motor vehicle is used, note the 
exact distance on the odometer from the nearest town to the clay site. A 
photograph of the clay site will help in finding the right site again later.
%-------------------------------------------------------------------------------
\subsubsection{Probe Digging}
In areas where initial survey and testing indicates deposits of suitable clay, 
holes should be dug in a regular grid in order to ascertain the size of the 
deposit. Initially, holes should be dug with a grid distance of 50 m. and where 
the best quality clay is discovered, the distance can be reduced to 15 m. It is 
worthwhile to ensure that the deposit is large enough to supply the planned 
production for a long time. The holes can be dug with a spade, but if a hole is 
more than 2 m deep the sides of the hole should be supported by planks.

A bucket auger (fig~\ref{fig}) is a very useful tool for taking samples. One or 
two people rotate the auger, which drills its way into the soil. The shaft can 
be extended so that samples can be taken from depth of 5 m or more. The bucket 
auger can be made at a local machine shop.
%-------------------------------------------------------------------------------
\subsubsection{Map}
A map should be made of the whole grid area, and on the map the probe holes are 
marked with a number, thickness of overburden and depth of clay layer.

The map is drawn with the help of a few fixed features, like trees and large 
stones (fig~\ref{fig}). The distance between three fixed points is measured as 
accurately as possible. The map is made in a scale of 1:100 (1 cm on the map 
represents 100 cm (1 m) in reality) or 1:200 (1cm=2m). The three fixed features 
are then marked on the map like on fig~\ref{fig}. The location of the test 
holes is measured and marked on the map according to its distance from the 
fixed points. Each test hole is given a number, which is marked by the hole 
itself, on the test sample label and on the map.

After testing of all the samples has pinpointed the best area, a more detailed 
plan (e.g. scale 1:50) of that area should be drawn, showing the depth of the 
various layers of top soil, clays and possibly other materials.
%-------------------------------------------------------------------------------
\subsection{Economy}
After having established the quality of the raw clay, the approximate size of 
the deposit and the thickness of the overburden the final decision of whether 
or not to start mining the clay remains to be done. Many factors control the 
economy of mining clay:
%-------------------------------------------------------------------------------
\begin{itemize}
\item The distance from the deposit to a suitable road, and the cost of 
transport to the workshop. It may be necessary to construct a small track from 
the deposit to the nearest road.
\item The cost of removing the overburden com- pared to the amount of clay 
underneath.
\item The cost of renting or buying land.
\item The quality of the clay. If the raw clay contains large amounts of sand, 
it may be necessary to wash the clay at the mine in order to reduce the cost of 
transport.
\end{itemize}
%-------------------------------------------------------------------------------
The total cost of opening up the mine, and the cost of digging and transporting 
the clay as listed above should be calculated as cost per kilogram of clay. 
This cost per kilogram is then compared with the cost of possible alternative 
sources of clay.
%-------------------------------------------------------------------------------
\section{Clay Mining}
\subsection{Before Mining}
The clay you intend to mine may have been deposited in an ancient lake as shown 
in fig~\ref{fig}. It may have taken hundreds of thousands of years to fill the 
lake 
with sediments, and during that period variations in climate, course of rivers, 
etc., caused the layers of sediments to vary. Each layer may contain its own 
type of clay or sand and the thickness of each layer may vary considerably.
%-------------------------------------------------------------------------------
\subsubsection{Clay Layers}
Before you start to dig the clay, expect it to be limited to certain layers. 
Today these layers will often be positioned horizontally in the same manner as 
they were laid down (fig~\ref{fig}). But they may also have been turned upside 
down by later folding of the landmasses and could be positioned as suggested in 
fig~\ref{fig} and fig~\ref{fig}. The digging of probe holes as mentioned above 
should 
indicate how the clay layers are positioned.
%-------------------------------------------------------------------------------
\subsubsection{Overburden}
First the top soil or overburden has to be removed and piled away from the clay 
pit where no future clay digging is planned. Care should be taken to avoid 
mixing top soil with the clay. If the overburden is several meters thick, it 
may be worthwhile to hire a bulldozer to clear away enough top soil for several 
years' clay mining.

In some cases, the overburden is too thick to be removed, and underground 
mining must be considered. This method is especially tempting if the clay is 
situated on a riverbank, on a slope or on an escarpment so that horizontal 
shafts can be dug. That will save the tedious task of removing the overburden, 
but this advantage may be offset by the extra cost of using lumber supports for 
keeping the walls of the underground shafts from collapsing. Underground mining 
of clay in vault shaped shafts without supports is often seen but is dangerous.
%-------------------------------------------------------------------------------
\subsubsection{Digging Tools}
Commercial mining of clay on a large scale in industrialized countries uses 
heavy machinery (fig~\ref{fig}). In most other places manual methods are more 
economical.

Manual digging is done by spade or hoe. A shovel is no good for breaking up 
clay, but it is useful when loading the clay into a wheelbarrow or truck. A 
wheelbarrow is used for bringing clay from the pit to a vehicle.
%-------------------------------------------------------------------------------
\subsubsection{Supervision of Clay Winning}
While digging, the worker should sort out roots, limestone, rocks and other 
unwanted material. The digging should always be supervised by an experienced 
person who can judge the quality of the raw clay. As the working of the clay 
pit progresses it may reach layers of inferior clay. The supervisor should 
regularly test the quality of the clay using the simple methods described 
above. Production of first class pottery demands raw materials of 
consistent and uniform quality. A potter adjusts production methods to the 
clay, and if it suddenly changes its behavior, it may ruin the production. If, 
for example, the clay becomes more plastic it may cause pots to crack during 
drying.

Even within the same layer of clay the composition of the raw clay may differ. 
Therefore, it is prudent to dig clay from several levels or locations at the 
same time, and to mix the material before loading it for transport. A worker 
can dig 4 to 8 tons of raw clay per day, depending on conditions in the clay 
pit. If the worker is paid according to quantity (piece work) the quality of 
the clay may suffer unless the supervision is thorough. In the long run, it may 
be less costly to pay by the hour (time work) in order to get better quality 
clay.
%-------------------------------------------------------------------------------
\subsubsection{Safety}
Clay is normally extracted from open pits, which are much safer compared to 
underground mining. However, safety should also be a concern in open pits. Clay 
should not be dug from a vertical clay face higher than 2 m because a large 
portion of the face could break loose and bury the worker. The digging of clay 
in deep pits should be done in benches as shown in fig~\ref{fig}. The 
overburden is removed away from the clay face being worked on, so that no top 
soil will get mixed with the clay itself. The benches on the clay face are made 
in steps about 1 m high and 0.5--1 m wide. The material from different levels 
can be thrown to the bottom of the pit for mixing, in order to even out 
variations in clay quality.
%-------------------------------------------------------------------------------
\subsubsection{Record}
A record should be kept of where in the pit the different batches of clay are 
extracted, so that sudden changes in the quality of clay can be traced to 
specific locations in the pit, and these can be avoided in the future. The 
movement of the digging area is recorded on the original map and is compared 
with the location of the original test holes. In this way, the supervisor can 
decide in which direction and to which depth to direct the digging, and he will 
be able to avoid clay beds of inferior quality that are shown on the test hole 
map.
%-------------------------------------------------------------------------------
\subsection{Stockpiling and Weathering}
\subsubsection{Planning}
Clay mining is impractical or impossible during rainy seasons, and in some 
areas of the world the ground freezes hard during winter. Therefore, it is 
necessary to extract sufficient clay during the dry season or during the summer 
to cover a whole year's production. That means you will have to plan your clay 
digging and production 0.5--1 year ahead, or you will run out of clay.
%-------------------------------------------------------------------------------
\subsubsection{Weathering}
Storing the clay in the open, and exposing it to the action of rain, sun or 
frost is called weathering. The alternate wetting and drying, or freezing and 
thawing, improves the plasticity of the clay by breaking it into smaller 
particles.

Weathering will reduce the content of possible organic matter in the clay, and 
this may have a bleaching effect on the raw clay. Weathering also washes out 
soluble salts. The clay should not be piled higher than 0.5 m. Clay may be 
weathered from a few months up to one year.
%-------------------------------------------------------------------------------
\subsubsection{Clay Storing}
When clay is received at the pottery it should be piled as shown in 
fig~\ref{fig}. Each truck or cart load is spread out in a thin layer covering 
the whole storage area. This makes horizontal layers.

The raw clay is collected from the clay bin by making vertical cuts in the 
pile, so that a little clay from each truck load is part of each batch of clay 
used in production. This procedure will ensure a more uniform clay quality.
%-------------------------------------------------------------------------------