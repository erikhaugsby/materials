\chapter{Nature of Clay}
A look into the chemical and physical properties of clay will help in 
understanding its behavior, and how to correct problems like cracking and 
warping which occur during drying or firing.
%-------------------------------------------------------------------------------
\section{Simple Clay Chemistry}
Chemistry is the science which describes what substances are made of and how 
they combine with each other. This science makes use of special names and 
symbols. Once learned, they are quite simple to understand.
%-------------------------------------------------------------------------------
\subsection{Elements}
An element cannot be broken down into more simple substances, and it consists 
of only one kind of atom. Oxygen (O) is the most common element on earth.
%-------------------------------------------------------------------------------
\subsection{Compounds}
A compound is composed of different elements bound together chemically. 

Water (\ce{H2O}) is a compound made up of two parts, or atoms, of hydrogen (H) 
and one part, or atom, of oxygen (O). Silica (\ce{SiO2}) is another compound 
and consists of one silicon atom (Si) and two oxygen atoms (O). This is the most
abundant material in the earth's crust.

Ceramic minerals are usually in the form of oxides: this means oxygen (O) is a 
part of the compound. Minerals are compounds.
%-------------------------------------------------------------------------------
\subsection{Mixture}
A mixture is a physical, not chemical, combination of compounds (and sometimes 
elements), and each compound remains chemically unchanged in the mixture. For 
example, a glaze made of feldspar, quartz and limestone is initially a mixture, 
but during firing a chemical combination takes place and the fired glaze 
changes into a compound.
%-------------------------------------------------------------------------------
\subsection{Chemical Symbols}
There are about 100 elements and each of these has a name and a chemical 
symbol, which is used as a shorthand name. 

Oxygen is written as capital O and hydrogen as H, whereas other symbols have 
two letters: silicon = Si and aluminum = Al. 

Compounds are written in a similar way, with capital letters marking the 
individual elements:
%-------------------------------------------------------------------------------
\begin{itemize}
\item $Water = \ce{H2O}$
\item $Salt = \ce{NaCl}$
\end{itemize}
%-------------------------------------------------------------------------------
The small number \textit{2} in \ce{H2O} indicates that there are two atoms of 
hydrogen for each atom of oxygen in water.

Formulas of complex ceramic materials are written as combinations of oxides 
with a high period (\ce{*}) between them to show they are chemically combined. 

For example, potash feldspar is written \ce{K2O*Al2O3*6SiO2}
%-------------------------------------------------------------------------------
\subsection{Chemical Reactions}
The formation of clay from feldspar can be written like this in chemical 
symbols:

\ce{K2O*Al2O3*6SiO2 + H2O -> Al2O3*2SiO2*2H2O + K2O + SiO2}

$Feldspar+Water \rightarrow Clay+Potash+Silica$

All materials are made of elements which are chemically bonded together. 
However, under certain conditions, a material may be changed into another by 
changes in the elements making up the material. Heat often provokes such 
chemical changes, and the production of quicklime by heating limestone to 
900\degree C is an example of this: 

\ce{CaCO3 -> CaO + CO2}

$Limestone+Heat \rightarrow Quickline+Carbon~dioxide$

Carbon dioxide (\ce{CO2}) mixes with air, and the remaining quicklime (CaO) is 
slaked with water and can then be mixed with sand to form a mortar for house 
construction. 

The trick is, that the mortar sets (becomes hard) when the calcium oxide (CaO) 
takes back carbon dioxide (\ce{CO2}) from the air and thereby regains the 
hardness of the original limestone (\ce{CaCO3}):

\ce{CaO + CO2 -> CaCO3}

$Soft~mortar+air \rightarrow Set~mortar$
%-------------------------------------------------------------------------------
\section{Chemical Changes in Clay Crystals}
%-------------------------------------------------------------------------------
\subsection{Kaolinite}
There are several types of clay minerals, so in individual clays the clay 
particles or crystals may differ. The clay mineral found in kaolin clay is 
called kaolinite and its formula is \ce{Al2O3*2SiO2*2H2O}.

This shows that for each part of alumina there are two parts of silica and two 
parts of water. The water (\ce{H2O}) of the clay mineral is not the physical water 
added to the clay to make it plastic, but chemical water existing within the 
kaolinite crystal itself.

When kaolin clay is fired, several changes occur within the clay crystal, as 
shown below. (It is, however, not necessary to remember these chemical 
reactions in detail. They serve here as illustrations of chemical changes 
taking place in the fired clay.)
%-------------------------------------------------------------------------------
\subsubsection{100--200\degree C}
Physical water evaporates.
%-------------------------------------------------------------------------------
\subsubsection{450--600\degree C}
The chemical water in kaolinite is released and steam can often be seen coming 
out of the chimney at this temperature. 

\ce{Al2O3*2SiO2*2H2O -> Al2O3*2SiO2 + 2H2O}. 

The release of the chemical water causes a weight loss of 13.95\% and the 
kaolinite crystals are permanently changed. This is called the ceramic change, 
in which the clay loses its plasticity forever.
%-------------------------------------------------------------------------------
\subsubsection{950\degree C}
A new crystal is formed at this temperature and the process is:

\ce{2(Al2O3*4SiO2) -> 2Al2O3*3SiO2 + SiO2}. 

One part of silica is released and adds to the free silica in the clay body. 
Free silica may already be present in the form of sand.
%-------------------------------------------------------------------------------
\subsubsection{1100--1400\degree C}

Gradually the clay changes into a crystal called mullite:

\ce{2Al2O3*3SiO2 -> 2(Al2O3*SiO2) + SiO2 -> 3Al2O3*2SiO2 + SiO2} 

($Pseudo-mullite \rightarrow Mullite$) 

More silica is released during mullite formation and the alumina content 
increases. Mullite crystals are long and needle shaped, and form a lattice 
structure which reinforces the clay body in much the same way as iron bars 
reinforce concrete structures. The silica is released in the form of 
cristobalite crystals which may cause dunting on fast cooling.
%-------------------------------------------------------------------------------
\subsection{Montmorillonite}
There are several other types of clay minerals, but we shall only discuss 
montmorillonite, which is often present in native clay along with kaolinite.

The montmorillonite mineral has this formula: \ce{Al2O3*4SiO2*H2O}

This mineral contain 4 silica (\ce{SiO2}) for each alumina (\ce{Al2O3}) which 
is twice as much compared to the amount of silica in kaolinite. Its crystal 
structure is also different from kaolinite, and it easily breaks into smaller 
particles. That makes the clay extremely plastic and gives it a soapy feeling. 

An addition of 1\% pure montmorillonite to a clay body may improve its 
plasticity as much as an addition of 10\% of ball clay. 

Bentonite is a primary montmorillonite clay mined in the U.S.A., but the term 
bentonite is often used for other commercial montmorillonite clays as well.

The release of free silica, takes place in montmorillonite above 950\degree C, 
but almost double the silica is released, compared to kaolin. Therefore, clay 
bodies with high amounts of montmorillonite contain a high percentage of free 
silica after firing, which may cause the ware to crack during cooling.
%-------------------------------------------------------------------------------
\section{Physical Nature of Clay}
\subsection{Shape and Size of Clay Minerals}
The clay crystals of kaolinite are shaped as flat hexagonal flakes 
(fig~\ref{fig}). They are extremely small and can only be seen with the help of 
an electron microscope. Each crystal contains thousands of layered sheets 
stacked on top of one another as in a pack of playing cards. The sheets are 
only loosely bonded together and they easily break into thinner flakes, which 
retain their hexagonal shape.
%-------------------------------------------------------------------------------
\subsection{Water of Plasticity}
Plasticity can be defined as the property of clay that enable it to be shaped 
without cracking and keep its new shape. This property is only found in clay.

Clay owes its plasticity to its thin plate like particles. When the clay is in 
a stiff plastic state, a thin film of water surrounds each clay particle. This 
film of water acts as a lubricant and enables the particles to slide past one 
another when the clay is formed, but the particles stick to one another and 
retain the shape once forming stops.

When more water is added to the plastic clay the clay particles start to move 
more freely and cannot hold onto one another as before. The clay becomes very 
soft and cannot retain its shape. After adding more water the clay becomes 
liquid, and in this state it is called a slip.
%-------------------------------------------------------------------------------
\subsection{Particle Size}
Plasticity, or the ability of the clay particles to hold onto one another, is 
directly related to the size of the clay particles. The smaller they are, the 
greater the total surface area and the more there is to hold onto.

A clay with large particles cannot pass our rope test, whereas a fine plastic 
clay can be bent without breaking. Each of its fine particles needs only to 
move a little to accommodate the bending, whereas the particles of the coarser 
clay have to move so far that they break apart.
%-------------------------------------------------------------------------------
\subsection{Electrical Charge}
Clay particles behave like small magnets, which attract each other when they 
have opposite polarity (North-South or plus-minus) but repel each other when 
they have the same polarity. The polarity of the particles depends on the 
non-clay materials. When the clay particles repel each other, the plasticity of 
the clay is low; whereas when they attract each other, the plasticity is high 
and more water is needed to make the clay soft.
%-------------------------------------------------------------------------------
\subsection{Souring}
Many sedimentary clays contain carbon (decayed vegetable matter), which make 
the clay sour (acid) during storage. The acid polarizes the clay particles so 
they attract each other, thereby increasing the plasticity of the clay. Adding 
calcium to a clay has a similar effect.

Additionally, aging clay allows bacteria to produce colloidal gel, a sticky 
slippery substance that adds to plasticity.
%-------------------------------------------------------------------------------
\subsection{Casting Slips}
A typical clay needs a 100\% addition of water to make it into a slip. For slip 
casting it is desirable to have as little water as possible, in order to reduce 
problems of shrinkage and wet plaster molds. Addition of washing-soda and 
water-glass (sodium silicate) changes the electrical polarity of the clay 
particles so that they repel each other and as little as 50\% addition of water 
will make the clay into a slip.
%-------------------------------------------------------------------------------
\subsection{Strength}
A clay containing very fine particles will collapse under its own weight during 
forming, because the particles slide too easily. Addition of coarser particles 
will give ``bone'' to the plastic clay by preventing the fine particles from 
sliding excessively. The additive can be sand, grog or a coarse clay like 
kaolin.
%-------------------------------------------------------------------------------
\subsection{Effect of Clay Preparation}
Clay crystals tend to cling together in lumps, that behave like large 
particles. By blunging the clay, especially in a high-speed blunger, these 
lumps can be broken down. Prolonged storage (also called aging) of the plastic 
clay under moist conditions gives the water time to penetrate the lumps of clay 
crystals and surround the individual crystals with its lubricating film. Water 
helps to break down the individual crystals, and so furthers the plasticity of 
the clay. Kneading and pugging brings the clay particles into closer contact, 
and helps to remove air pockets. This improves the strength and plasticity of 
the clay, and prevents forming or firing problems due to trapped air.
%-------------------------------------------------------------------------------
\section{Drying}
After forming, the next step in pottery production is drying. All potters have 
experienced warping or cracking during drying, so let us look at the causes of 
these problems.
%-------------------------------------------------------------------------------
\subsection{Surrounding Air}
In the drying process, all the lubricating water (also termed water of 
plasticity) has to get out of the clay and into the surrounding air. When the 
water content of the clay is equal to the surrounding air the process of drying 
stops.
%-------------------------------------------------------------------------------
\subsection{Drying Shrinkage}
Fig~\ref{fig} shows 4 stages of a clay from forming condition to bone dry. 
``A'' shows the clay in its plastic state; all particles are surrounded by 
water (shown as small dots) and they easily slide when the clay is formed. When 
the clay is left to dry, the water moves from within the clay to its surface 
through the pores between the clay particles.

As the water leaves the spaces between the particles, they move closer together 
and will finally touch one another as shown in ``B''. The clay shrinks as the 
water disappears, so clearly the more water a clay requires to become workable, 
the more it will shrink on drying. That also means that the more plastic a clay 
is, the more it shrinks. At this stage, termed leather hard, there is still 
plenty of water left between the clay particles, but since these are already in 
contact no more shrinkage will take place.
%-------------------------------------------------------------------------------
\subsection{Pore Water}
The water between the clay particles will continue to move out of the clay 
until the moisture content is the same inside and outside the clay. The 
remaining water is called pore water, and the finer the clay particles, the 
higher the amount of pore water. Only when the clay is heated to above 
100\degree C will the last pore water escape.

The pore water may be as much as 5\% of the clay weight, and it is therefore 
important that the initial heating of the kiln is done very slowly, so that 
this pore water can escape before it turns to steam at 100\degree C. Steam will 
crack the pot or cause pieces of it to explode.
%-------------------------------------------------------------------------------
\subsection{Rate of Drying}
All potters know that clay dries faster in dry, warm and windy weather, and 
that the rate of drying can be slowed down by covering the clay with plastic 
sheets or wet cloth. Clay ware must dry evenly so that it shrinks evenly. A 
handle on a cup tends to dry faster than the cup itself, and the different rate 
of shrinkage will produce a crack in the handle unless care is taken to let the 
whole cup dry slowly.

Water from the core of the clay travels through the thickness of the clay to 
the surface, through all the small gaps between the clay particles. Therefore, 
clay with very fine particles dries slowly, and coarse clay dries faster. 
Fig~\ref{fig} shows how the water in a very fine clay cannot pass through the 
outer layer of the clay, which has already dried and closed the gaps. This can 
be corrected by ``opening up'' the clay: i.e. adding grog, sand or another 
coarse grained clay (fig~\ref{fig}). 

Ware with thin walls dries quickly and evenly. Thick-walled designs are more 
likely to warp or crack. They should be dried very slowly, and additions of 
sand or grog up to 20\% are very helpful.
%-------------------------------------------------------------------------------
\section{Particle Orientation}
\subsection{Unstable Particles}
Playing cards standing on their edge are very unstable arrangements and clay 
minerals, having a rather similar shape, behave in the same way. When pressure 
is applied, clay particles position themselves with their flat sides facing the 
pressure.
%-------------------------------------------------------------------------------
\subsection{Particle Orientation}
An example of this is shown in fig~\ref{fig}. A clay with its particles 
randomly positioned is left to mature for a long time. Gradually the force of 
gravity causes the particles to orientate themselves with their flat side 
facing the pull of gravity.
%-------------------------------------------------------------------------------
\subsection{Differential Shrinkage}
Such alignment of particles produces higher drying shrinkage in one direction, 
since more particles (and more water) are stacked in that direction. The 
additional water causes greater shrinkage during drying, and the shrinkage in 
one direction may be several times greater than in the other direction 
(fig~\ref{fig}). This difference may cause problems in drying and firing.
%-------------------------------------------------------------------------------
\subsection{Throwing}
When throwing a pot on the wheel, pressure is applied to the wall of the pot 
from both sides and the clay particles will position themselves parallel to the 
wall (fig~\ref{fig}). In forming the bottom of the pot, pressure should be 
applied towards the wheelhead, while moving the fingers from outside to the 
centre. Otherwise, the particles will not be aligned, and the bottom will crack 
(fig~\ref{fig}).
%-------------------------------------------------------------------------------
\subsection{Extrusion}
When forming clay by extrusion, particle orientation takes place when the clay 
is forced through the die (fig. 43-5). A screw extruder produces another 
problem, called lamination, by the pressure from its screw blades. In the 
extruded column of clay a spiral lamination is formed. This may cause products 
to warp or crack during drying or firing.
%-------------------------------------------------------------------------------
\subsection{Slip Casting}
In a slip casting mould, the suspended clay particles are sucked towards the 
inner wall of the plaster mould, and they align themselves with their flat 
sides towards the mould face (fig. 44-1). If the design of the cast has sharp 
corners, the particle orientation (and thereby the direction of shrinkage) will 
be at right angles to each other and the pot may crack here during drying.
%-------------------------------------------------------------------------------
\subsection{Kneading}
During prolonged storage the clay particles position themselves according to 
the pull of gravity and one purpose of kneading (wedging) the clay before 
forming is to break up this particle orientation.
%-------------------------------------------------------------------------------
\subsection{Firing}
During firing, shrinkage also takes place, and particle orientation may create 
problems of warping similar to those mentioned for drying.
%-------------------------------------------------------------------------------
\section{Firing}
The clay body goes through a number of stages during firing.
%-------------------------------------------------------------------------------
\subsubsection{$<$ 120\degree C: Water Smoking}
The water of plasticity evaporates first and then the pore water. Rapid 
increase of temperature will build up steam pressure, and may crack the clay.
%-------------------------------------------------------------------------------
\subsubsection{220\degree C: Cristobalite expansion}
Cristobalite is created from silica at temperatures above 900\degree C. When 
the clay is fired a second time it will expand nearly 3\% at 220\degree C. On 
cooling, cristobalite shrinks again. Rapid cooling at this temperature may 
crack ware.
%-------------------------------------------------------------------------------
\subsubsection{350--600\degree C: Ceramic Change}
As described above, the chemically bound water of the clay crystal is released. 
The clay is very fragile and porous at this point. The clay particles are held 
together by a sort of ``spot welding'' at the points of contact. This process 
is called sintering (fig~\ref{fig}).
%-------------------------------------------------------------------------------
\subsubsection{573\degree C: Quartz Expansion}
The quartz crystal (\ce{SiO2}) expands suddenly and will shrink again at this 
point during cooling (about 1\%). The clay structure during heating is still 
open enough to accommodate this change, but if cooling is too rapid, the ware 
may crack.
%-------------------------------------------------------------------------------
\subsubsection{500--900\degree C: Oxidation}
Organic matter in the clay is burned out. If the clay has a black core after 
firing, then this stage of firing was done too fast. When the rise of 
temperature is very rapid, the surface may vitrify before the carbon dioxide 
inside the clay has escaped, and the entrapped gas will bloat the clay at a 
later stage of firing. ``Bloating'' is seen as bubbles or voids, which occur 
inside the clay and on the surface.

Limestone (\ce{CaCO3}) gives off its carbon dioxide (\ce{CO2}) at 825\degree C.
%-------------------------------------------------------------------------------
\subsubsection{$>$ 900\degree C: Vitrification}
At this temperature the soda and potash in the clay will start to form a glass 
by combining with the free silica. As the temperature rises, more and more 
glass will be formed, involving materials like limestone, talc and iron oxide. 
The melted glass will gradually fill the pores between the clay particles as 
shown in fig~\ref{fig}. This vitrification process also causes the clay to 
shrink.
%-------------------------------------------------------------------------------
\subsection{Firing Shrinkage}
Firing shrinkage is a simple indication of how much a clay is vitrified. 
Another indication is how much water the fired clay can absorb. Vitrified clay 
has more glass filling its pores, so it absorbs less water.
%-------------------------------------------------------------------------------
\subsection{Glass Melt}
The melted glass formed in clay bodies is normally stiff, and will not cause 
the clay to collapse suddenly. Feldspar produces a stiff glass that allows for 
a long firing range. Limestone, on the other hand, only becomes a strong liquid 
flux when fired to above 1100\degree C. This will cause the ware to collapse 
suddenly.

Glass of vitrification produces a strong finished body. But if the body is 
fired too high it will lose strength and become brittle.
%-------------------------------------------------------------------------------
