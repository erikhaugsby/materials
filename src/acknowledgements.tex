\chapter*{Acknowledgments}
\section*{This PDF Document}
\subsection*{Typesetting}
This document was typeset by Erik Haugsby from the source material provided 
via the CD3WD:
%-------------------------------------------------------------------------------
\begin{verbatim}
http://www.fastonline.org/CD3WD_40/CD3WD/APPRTECH/G41CLE/EN/B159_2.HTM
\end{verbatim}
%-------------------------------------------------------------------------------
This document attempts to replicate the source material and present it in a 
more accessible structure.

As the source material is freely available online and the cost of a print 
version of this text is prohibitively high, I believe this document is an 
important contribution to the collective knowledge of potters. 

It improves upon the HTML version of the information by consolidating all 
sections into a single document, which is both viewable on the computer and can 
also be printed. Furthermore, it introduces clickable links/references between 
sections and tables.
%-------------------------------------------------------------------------------
\subsection*{Formatting: Changes, Errors, Omissions}
I have attempted to preserve the original text and formatting. 

Some changes to formatting, especially of tables, was unavoidable due to the 
LaTeX typesetting.

Some corrections were made to spelling and grammar.

Unfortunately, no images are available in the online version of this document. 
Inline references to images have been maintained, but will always appear as 
\textbf{??}. Appendices are also missing.

The possibility of minor or unavoidable changes in layout and formatting, as 
well as unintentional errors in transcribing the original text, cannot be 
excluded.

Should you find any errors or omissions, I would greatly appreciate you either 
notifying me \href{mailto:e@erikhaugsby.com}{by email}, or initiating a pull 
request via Git. 

Link to GitHub project: 
\href{https://github.com/erikhaugsby/materials/}{https://github.com/erikhaugsby/materials/}
%-------------------------------------------------------------------------------
\subsection*{Citation}

Norsker, Henrik, and James Danisch. \textit{Clay Materials - for the 
  Self-reliant Potter: A Publication of Deutsches Zentrum F\"{u}r 
  Entwicklungstechnologien - GATE, a Division of the Deutsche Gesellschaft 
  F\"{u}r Technische Zusammenarbeit (GTZ) GmbH.} Braunschweig: Vieweg, 1990.

ISBN 10: 3528020571

ISBN 13: 9783528020576
%-------------------------------------------------------------------------------
\newpage
%-------------------------------------------------------------------------------
\section*{Acknowledgments}
Potters and colleagues in Bangladesh, Burma, Denmark, India, Nepal, Tanzania 
and Thailand have helped me in collecting the material and experiences 
presented in this book. Shailendra Maharjan has produced most of the drawings. 
James Danisch has contributed valuable suggestions and has corrected my English.
%-------------------------------------------------------------------------------
\section*{The Author}
\subsection*{Henrik Norsker} 
Henrik Norsker has been making pottery since 1970. He left his pottery workshop 
in Denmark in 1976 to help with establishing a pottery school in a village in 
Tanzania. Since then he has continued working with promotion of modern pottery 
in developing countries. Besides Tanzania, he has worked for ceramics projects 
in Bangladesh and Burma. He is presently working for a ceramics project in 
Nepal.
%-------------------------------------------------------------------------------
\subsection*{The Deutsches Zentrum f\"{u}r Entwicklungstechnologien}
Deutsches Zentrum f\"{u}r Entwicklungstechnologien-GATE

Deutsches Zentrum f\"{u}r Entwicklungstechnologien--GATE--stands for German 
Appropriate Technology Exchange. It was founded in 1978 as a special division 
of the Deutsche Gesellschaft f\"{u}r Technische Zusammenarbeit (GTZ) GmbH. GATE 
is a 
centre for the dissemination and promotion of appropriate technologies for 
developing countries. GATE defines ``Appropriate technologies'' as those which 
are suitable and acceptable in the light of economic, social and cultural 
criteria. They should contribute to socio-economic development whilst ensuring 
optimal utilization of resources and minimal detriment to the environment. 
Depending on the case at hand a traditional, intermediate or highly-developed 
can be the ``appropriate'' one. GATE focusses its work on the key areas:
%-------------------------------------------------------------------------------
\begin{itemize}
\item Dissemination of Appropriate Technologies: 

Collecting, processing and disseminating information on technologies 
appropriate to the needs of the developing countries: ascertaining the 
technological requirements of Third World countries: support in the form of 
personnel, material and equipment to promote the development and adaptation of 
technologies for developing countries.

\item Environmental Protection:

The growing importance of ecology and environmental protection require better 
coordination and harmonization of projects. In order to tackle these tasks more 
effectively, a coordination center was set up within GATE in 1985.
\end{itemize}
%-------------------------------------------------------------------------------
GATE has entered into cooperation agreements with a number of technology 
centres in Third World countries.

GATE offers a free information service on appropriate technologies for all 
public and private development institutions in developing countries, dealing 
with the development, adaptation, introduction and application of technologies.

Deutsche Gesellschaft f\"{u}r Technische Zusammenarbeit (GTZ) GmbH

The government-owned GTZ operates in the field of Technical Cooperation. 2200 
German experts are working together with partners from about 100 countries of 
Africa, Asia and Latin America in projects covering practically every sector of 
agriculture, forestry, economic development, social services and institutional 
and material infrastructure. The GTZ is commissioned to do this work both by 
the Government of the Federal Republic of Germany and by other government or 
semi-government authorities.

The GTZ activities encompass:
%-------------------------------------------------------------------------------
\begin{itemize}
\item appraisal, technical planning, control and supervision of technical 
cooperation projects commissioned by the Government of the Federal Republic or 
by other authorities

\item providing an advisory service to other agencies also working on 
development projects

\item the recruitment, selection, briefing, assignment, administration of 
expert personnel and their welfare and technical backstopping during their 
period of assignment

\item provision of materials and equipment for projects, planning work, 
selection, purchasing and shipment to the developing countries

\item management of all financial obligations to the partner-country.
\end{itemize}
%-------------------------------------------------------------------------------
Deutsches Zentrum f\"{u}r Entwicklungstechnologien--GATE:

Deutsche Gesellschaft f\"{u}r Technische Zusammenarbeit (GTZ) GmbH

P.O. Box 5180

D-65726 Eschborn

Federal Republic of Germany


Tel.: (06196) 79-0

Telex: 41523-0 gtz d

Fax: (06196) 797352

