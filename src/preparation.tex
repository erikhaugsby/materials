\chapter{Clay Washing and Clay Body Preparation}
\section{Washing at the Pit}
After extracting the clay from its pit, it is normally brought directly to the 
pottery for further treatment. However, some raw clays contain so much sand 
that it is more economical to dispose of the sand at the clay pit, thus 
avoiding the cost of transporting this unwanted material. The true clay content 
found in primary clay deposits may be only 15\% or less. The whiteness or the 
refractoriness of such clays may still make it profitable to mine them.
%-------------------------------------------------------------------------------
\subsection{Commercial Kaolin Mine}
In some commercial kaolin mines, clay is extracted by washing the clay face in 
an open kaolin pit with a high-pressure water-jet. The fine clay is carried 
away with the water leaving behind most of the coarse materials. In older 
mines, the resulting clay slurry is run through troughs, where the non-clay 
materials settle, after which the clay slurry is pumped into settling tanks. 
After siphoning off the clear water, the clay slip is de-watered in filter 
presses and dried in ovens. In more modern works, the trough type settling 
tanks have been replaced by centrifuges.

Smaller clay works cannot afford such machinery. Instead the raw clay is mixed 
with water, and stirred either manually in ponds or in a washmill.
%-------------------------------------------------------------------------------
\subsection{Washing Ponds}
Next to the clay pit and close to a source of water two or more shallow ponds 
are dug, measuring about 4 m by 2 m and 0.5 m deep. The sides of the ponds can 
be lined with bricks or simple wickerwork plastered with clay.

The pond is half filled with water and the raw clay is added until the pond is 
nearly full. The raw clay is then stirred with a shovel, until all the clay is 
separated from the sand. With dry sandy clays this may take less than 20 
minutes. Finer clays need longer stirring and for very fine ones it may be 
necessary to let the clay soak for a day. However, clay that fine is in no need 
of having its sand removed.

When all the clay is suspended in water and the material at the bottom of the 
pond is only sand (without a clayey feeling), the clay slurry can be 
transferred to a settling pond. If possible the settling pond should be located 
at a slightly lower level so that the clay slurry can run in by itself. 
Otherwise, the clay slurry is transferred by bucket, which is filled in a small 
pit connected to the stirring pond with a pipe. In this way the bucket will not 
disturb the settled sand in the stirring pond.

When a very pure clay is desired, the clay slip is led through several settling 
pits before filling the final settling pond. During the slow flow through the 
intermediate pits, the fine sand particles will settle and only the much finer 
clay particles will flow on.

After the clay has settled in the settling pond, the clear water on top can be 
transferred by pump or by bucket, back to the stirring pond, which first has 
been emptied of its sand. The bucket or pump intake should not be dipped into 
the settling pond since that would stir the settled clay. Instead the surplus 
water is led to a small pit next to the settling pond, and water is taken from 
here. For each stirring pond several settling ponds are needed. The clay is 
then left in the settling pond until it is stiff enough to be removed for 
further drying.
%-------------------------------------------------------------------------------
\subsection{Washmill}
For large quantities of clay an animal powered washmill is useful 
(fig~\ref{fig}). The circular tank is half filled with water and raw clay, 
preferably dry and without large lumps, is added while stirring, until the tank 
is almost full. After stirring for 1--2 hours (depending on the properties of 
the clay), the clay will be suspended in the water, while stones and coarse 
sand settle at the bottom. The clay slurry is then run into settling ponds.

The washmill can also be operated continuously; first it is filled as described 
above and when the clay and sand have been separated, more raw clay is added 
gradually. The added raw clay will sink to the bottom, where it is worked by 
the harrows, and an equal amount of clay slurry will run off at the upper 
outlet. Fresh water is added from a pit with an inlet at the bottom of the 
tank. After operating the washmill continuously for some days, sand and stones 
accumulated at the bottom of the tank has to be cleaned out.
%-------------------------------------------------------------------------------
\subsubsection{Sand Separator}
If the fine sand content of the clay is not desired, the slurry is run (on its 
way to the settling tanks) either through a fine mesh sieve, a grooved tray 
(fig~\ref{fig}), a series of settling tanks (fig~\ref{fig}) or a sand separator 
(fig~\ref{fig} and fig~\ref{fig}).
%-------------------------------------------------------------------------------
\subsubsection{Washmill Construction}
The washmill is built of bricks laid with cement mortar. The centre section 
supports the beam that turns the harrows. The beam turns in simple bearings, 
that can be made of hardwood which is kept well greased. A sluice system with 
adjustable outlet levels can be fitted with a grate to catch roots and other 
organic material. The harrows should be made of iron or of wood reinforced with 
iron. Fig~\ref{fig} shows how an extra set of harrows hinged onto the main 
harrows 
improves the disintegrating action. The tank should not be more than 1 m deep, 
and increased capacity is achieved by widening the diameter of the circular 
tank. A tank with a diameter of 4.5 m can wash at least 6 ton of raw clay at a 
time.

The washmill can be powered by one or two persons or by a draught animal. A 
motor could also power the washmill, but since a rather high gear ratio is 
needed, the washmill becomes much more costly. Alternatively, the raw clay can 
be washed in a high speed blunger.
%-------------------------------------------------------------------------------
\section{Clay Body Preparation}
Two different methods are used for preparing the clay for production: dry and 
slop. Each method has many variations, and the right choice of techniques and 
machinery depends very much on the nature of the clay and on the type of ware 
to be produced. Therefore, before deciding on any method, the clay should first 
be tried out thoroughly--from preparation of clay to forming, glazing and 
firing.
%-------------------------------------------------------------------------------
\subsection{Dry Method}
After weathering, the clay is dried completely. If the clay can be used as it 
is, without having sand or stones removed, the dried clay is wetted directly 
and left until it has a plastic consistency. Then it is kneaded manually or in 
a pugmill. fig~\ref{fig} shows such a system with dry clay stored in the back, 
a 
slaking pit in front, plastic clay covered under wet cloth ready to be pugged.

In must cases, however, sand and other impurities have to be removed. The dried 
clay is then first pulverized either manually by a lever hammer, or with the 
help of a hammer mill or pin mill.
%-------------------------------------------------------------------------------
\subsubsection{Lever hammer}
The lever hammer is operated by two persons. One steps on the short lever, 
thereby lifting the long lever onto which a heavy iron or wooden hammer is 
fixed. This falls on the clay, which is fed to the hammer by the second person, 
who may also screen the powdered clay.

In many countries, the same traditional machine is used for rice hulling.
%-------------------------------------------------------------------------------
\subsubsection{Hammer mill}
A hammer mill or plate mill of the type used for grinding grain may be cheaper 
to operate compared to the lever hammer. It can be powered either by an 
electric motor or small diesel engine. In the hammer mill, a number of small 
exchangeable hammers are mounted on a rotating disk. The hammers disintegrate 
the material by impact until the particles are small enough to pass through a 
curved screen. The screens can be changed, so that coarse material like grog 
can be produced as well.

The rotation of the hammers produces suction at the centre and pressure at the 
perimeter. Therefore, the inlet of material should be at the centre the point 
of suction. The pressure helps to blow the ground material through the screen. 
The outlet should lead into a big cotton bag that retains the clay but lets the 
air through. Some hammer mills have the inlet at the perimeter, and that 
produces a lot of dust, which is a health hazard to the operator. The dust 
nuisance can be reduced by fixing an outlet with a long tube on top of the mill 
casing to release air pressure.
%-------------------------------------------------------------------------------
\subsubsection{Pin mill}
This mill has a rotor fitted with beaters which rotate between stationary pins. 
The material is fed through a hopper to the centre of the mill. The rotating 
beaters hit the material and fling it outward where it hits stationary pins. 
When it is fine enough, it passes the sieve surrounding the rotor.

The pin mill is suitable for slightly moist clay materials, which otherwise 
tend to clog the hammer mill screen.
%-------------------------------------------------------------------------------
\subsubsection{Screening}
After the clay has been pulverized, it can be mixed with other dry materials 
like sand, limestone, feldspar, kaolin, talc or grog (see p. 47). The mixture 
should be screened again, or put through the hammer mill an extra time in order 
to ensure proper mixing of the different materials.

For larger potteries a vibrating screen as shown in fig~\ref{fig} will be 
useful. 
The screen can be replaced so it can be used for different particle sizes.
%-------------------------------------------------------------------------------
\subsubsection{Wetting}
Water is then added to the clay, either by pouring water into a pit in the 
centre of the clay and leaving it to be absorbed, or by sprinkling an even 
amount of water onto about 5 cm thick layers of clay, which are then covered by 
subsequent layers. After soaking, the clay is kneaded either by foot or hand. 
The clay should then be left in a moist place for at least a week. This could 
be a clay cellar, or a clay pit covered with wet cloth and plastic sheets.

Prolonged storage will improve the plasticity of the clay and give it a stiffer 
consistency.
%-------------------------------------------------------------------------------
\subsubsection{Kneading}
After storage, the clay should be kneaded again, either manually or in a 
pugmill. A vertical pugmill can be constructed locally. An animal powered 
pugmill of a type often used in brickworks is shown at fig~\ref{fig}. The 
barrel is made from metal or wood and a vertical shaft is fitted with blades 
set at a slight angle, which forces the clay downward.
%-------------------------------------------------------------------------------
\subsubsection{Conclusion}
The advantage of the dry method is that little equipment is needed. Its 
drawbacks are that the workers are exposed to unhealthy clay dust, and that the 
clay only develops its full potential plasticity after prolonged storage. In 
some cases, troublesome impurities can only be removed by the slop method.
%-------------------------------------------------------------------------------
\subsection{Slop method}
The clay is made into a slip, other materials are added, and after screening 
the slip is dewatered until the clay has the right consistency. There are 
numerous variations of the slop method and a few examples are described below.
%-------------------------------------------------------------------------------
\subsection{Industrial slip house}
In industrial production of white ware, clay bodies are made up of several 
different clays, with additions of feldspar, quartz and other materials. 
Feldspar and rock quartz are first crushed to the size of gravel (2--4 mm) 
before further grinding in a ball mill. For that, one of the following machines 
is used.
%-------------------------------------------------------------------------------
\subsubsection{Jaw crusher}
This machine is used for the initial crushing of rocks (fig~\ref{fig}). Large 
lumps of rock are reduced to 15-25 mm pebbles. This initial reduction can also 
be done manually with a hammer. A jaw crusher is often used for producing grog.
%-------------------------------------------------------------------------------
\subsubsection{Roller crusher}
Crushing rollers are used for breaking down lumpy clay or shale. The rollers 
may be smooth or with grooves, and they rotate in opposite directions to each 
other. The space between the rollers can be adjusted.
%-------------------------------------------------------------------------------
\subsubsection{Pan grinder}
One or two heavy wheels made of granite or steel rotate on a pan of similar 
material and crush the pebble size material to sand (fig~\ref{fig}). In some 
cases, the pan is perforated. The pan grinder can also be used for crushing 
grog and for preparing clay bodies, especially granulated bodies for dust 
pressing of tiles.
%-------------------------------------------------------------------------------
\subsubsection{Hammer mill}
The hammer mill is widely used because it is so versatile. It is mainly used 
for softer materials like clay, limestone, talc and gypsum, but can also be 
used in an emergency for grinding feldspar and quartz, provided these have 
first been shattered by calcining above 600\degree C. The hard materials will 
quickly wear out the hammers and the sieve of the mill. ball mill: the final 
grinding takes place in a ball mill, which can grind materials up to coarse 
sand size. A ball mill is a hollow mild steel cylinder, lined with special 
bricks made of hard rock (like granite), or porcelain, or a thick rubber sheet. 
The ball mill is filled about 50\% with pebbles of flint or porcelain, 25\% 
material for grinding and 20\% water (measured by volume). The ball mill is 
rotated slowly, so that the pebbles constantly roll down the inner slope of the 
cylinder, and the material is ground by the rubbing action between the pebbles. 
(See appendix~\ref{fig}).

In large factories, each raw material is ground separately and then mixed in a 
blunger according to its slop weight. In smaller factories kaolin, feldspar and 
quartz are measured by dry weight and milled together. In that way the total 
ball milling time can be reduced by adding kaolin after the harder materials 
(quartz and feldspar) have been milled for some time.

A coarse screen fitted to the ball mill holds back the pebbles while the clay 
slip is poured out and led to a blunger through wooden troughs.
%-------------------------------------------------------------------------------
\subsubsection{Blunger}
In the blunger (mixing ark) plastic sedimentary clay (often ball clay) is added 
to the milled materials. The blunger shown at fig~\ref{fig} stirs the clay slip 
by rotating two sets of blades at 17 rpm (rotations per minute), and blunging 
time for ball clay is more than 10 hours. A high speed blunger with a single 
shaft propeller (fig~\ref{fig}) cuts blunging time to 1--2 hours and is now 
replacing the slower type.
 %-------------------------------------------------------------------------------
\subsubsection{Screening and de-magnetizating}
After blunging, the clay slip is screened through a fine mesh screen (80--150 
mesh per sq. inch). The screen is vibrated to prevent it from getting clogged. 
By fitting a coarser sieve mesh above the fine mesh screen the clogging problem 
can be reduced. 2-- and 3--deck screens are commonly used.

In production of white ware, the clay slip is passed through magnets that catch 
iron compounds, which would otherwise produce brown specks in the fired 
product. The magnets are either permanent magnets or electromagnets. Old 
loudspeaker magnets suspended in the blunger are adequate.
%-------------------------------------------------------------------------------
\subsubsection{Filter press}
The clay slip contains about 50\% water when being screened, and half of that 
must be removed before the clay has the right consistency for forming. In 
modern industries, dewatering is done in a filterpress. This consists of a 
series of frames which form chambers when fitted together. Each chamber is 
lined with a filter cloth, and the slip enters the chambers through holes in 
the frames. The filter cloth is hung over both sides of the frame and the two 
halves of the cloth are sewn together around the inlet hole. The frames are 
fitted together in one long row, and sealed by tightening a heavy screw.

The clay slip is then pumped into the filterpress under pressure ($7--10 
kg/cm^2$) and the water is forced out through the filter cloth. The water 
drains away through grooves in the frames.

Filtering time varies according to the particle size of the clay and the 
pumping pressure. Filtering of a coarse grained kaolin clay may take only two 
hours, whereas a highly plastic clay may take 8 hours. After the water has 
stopped dripping from the frames, the tightening screw is opened and stiff clay 
cakes are removed from between the frames and transferred to the clay storage.
%-------------------------------------------------------------------------------
\subsubsection{Compressed air pump}

A durable and simple filterpress pumping system can be made with the help of an 
air-compressor and a tank capable of handling pressures up to $10 kg/cm^2$. At 
the bottom of the tank a pipe connects to the filterpress inlet. After filling 
the tank with clay slip, compressed air is pumped into the tank. An adjustable 
pressure valve maintains the desired pumping pressure. When the clay in the 
filterpress is dewatered, the compressed air is shot off, the pressure in the 
tank is released and another batch of clay slip is loaded.

This system has several advantages over a conventional piston pump system:
%-------------------------------------------------------------------------------
\begin{itemize} 
\item The clay slip is forced through the filterpress under a constant 
pressure, thus reducing filtering time and lamination problems.
\item Maintenance cost is very low, since no moving parts come in contact with 
the abrasive clay slip.
\item The pump system is much cheaper and it can be made from locally available 
parts.
\item It uses very little energy.
\end{itemize}
%-------------------------------------------------------------------------------
The pressure tank is the most expensive part, but this cost can be reduced by 
using a smaller tank, which is then charged twice during one filterpress 
operation.
%-------------------------------------------------------------------------------
\subsubsection{Pugging}
The clay is then left to mature in the storage area for at least two weeks in 
order to improve its plasticity. However, the clay still requires hand or 
machine kneading to remove entrapped air and make the clay uniform.
%-------------------------------------------------------------------------------
\subsubsection{Kneading table}
A kneading table works the clay with horizontal and vertical rollers 
(fig~\ref{fig}). The upper horizontal rollers press the clay down, and 
afterwards the vertical rollers press the clay up. This alternate movement up 
and down squeezes entrapped air out of the clay and gives it a uniform 
consistency. The clay is laid in a circular wad on top of the kneading table, 
and each operation takes about 50 minutes.

The kneading table is mainly for bodies with low plasticity, like porcelain 
bodies, but is little used today.
%-------------------------------------------------------------------------------
\subsubsection{Pugmill}
The pugmill has replaced the kneading table in modern ceramics plants. It is a 
more costly machine, but it produces a better quality clay, especially if the 
clay is de-aired during pugging. The pug mill consists of a large cylinder with 
an axle running through its centre.

Sets of iron blades are attached to the axle and these both cut the clay and 
move it forward. Clay is fed from one end and pressed out through a mouth piece 
at the other end in a continuous column, that is cut up in convenient pieces 
ready for forming.

The flow of clay through the cylinder is not even. The clay at the centre moves 
at a different speed compared to the clay next to the cylinder wall, and this 
produces different densities in the clay body. That causes stress in the 
finished ware, which may warp during drying or crack during firing (these 
special cracks are called lamination cracks). Hand kneading after pugging is 
often required.

A de-airing pug mill takes the clay through a chamber where a strong vacuum is 
maintained by a vacuum pump. In this vacuum all entrapped air is sucked out of 
the clay, which greatly improves its plasticity. At the same time, de-airing 
reduces the problems of lamination stresses in the finished products.
%-------------------------------------------------------------------------------