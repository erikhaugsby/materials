\chapter{Clay Bodies}
\section{Raw Materials}
Only a few clays can be used as they are found in nature for pottery 
production. Most often, addition of other clays or materials like sand, 
limestone or feldspar is needed to produce a mixture suitable for specific 
forming and firing techniques. A mixture made up from different clays and 
materials is called a clay body.

Different methods of shaping demand different types of bodies. For the potter's 
wheel a good plastic clay is desirable, whereas much less plasticity is needed 
for pressing of tiles.

The ingredients of a clay body can be divided into four groups according to 
their function: plasticizer, filler, opener and vitrifier. The plasticizer is 
used for increasing the plasticity, the filler is the cheap bulk material, the 
opener promotes drying and reduces plasticity, and the vitrifier or flux is 
makes the body more fusible.

All the non plastic materials like feldspar, whiting etc. also work as openers 
of the body.
%-------------------------------------------------------------------------------
\subsection{Kaolin}
This is a primary clay and therefore rather coarse, with little plasticity. 
Kaolin opens up a plastic clay, so that it dries more easily. It is a 
refractory clay, increasing the melting point of the body, and it makes the 
body more white. The clay mineral in kaolin is kaolinite, which benefits 
bodies. Kaolin is a more costly material for clay bodies. Raw kaolin, i.e. 
unwashed kaolin containing high amounts of silica sand, is cheaper and 
beneficial if the clay body can tolerate the extra sand.
%-------------------------------------------------------------------------------
\subsection{Ball Clay}
This is a secondary clay with fine clay particles, which make it very plastic. 
Ball clay normally fires white, but the term is sometimes used for any highly 
plastic clay used to increase plasticity of a body. The white firing types 
increase maturing temperature of fusible clays and add whiteness. 

Many ball clays are grey or black when dug, indicating a high content of 
vegetable matter (carbon). 10\%--20\% addition of ball clay greatly improves 
the plasticity and green strength of a clay body. A white firing ball clay is 
normally expensive.
%-------------------------------------------------------------------------------
\subsection{Native Clay}
This simply means a secondary clay, firing to a red or buff colour. It makes up 
the bulk of the body for buff or dark firing earthenware and stoneware. 
Preferably, it should be dug close to the site of the workshop to keep its cost 
down. In the best of cases, it can be used without any additions. Adjusting the 
properties of a native clay can be done by adding kaolin for opening up the 
clay to reduce drying cracks, or by adding ball clay to improve its plasticity, 
as listed in fig~\ref{fig}. However, it is better and cheaper to try to find 
another 
native clay that can do the same job.
%-------------------------------------------------------------------------------
\subsection{Bentonite}
This is the commercial name for an extremely plastic clay also known as 
montmorillonite clay. It exists in smaller quantities in many natural clays, 
but in pure form it is used as a drilling fluid, in paint and chemical 
industries and for binding molding sand for iron casting.

In ceramics it is used to increase plasticity and green strength of clay 
bodies, and an addition of 1--2\% to a glaze helps to keep the glaze materials 
suspended. An addition of 1\% bentonite may increase plasticity of a clay body 
as much as 10\% ball clay.

When montmorillonite is present in larger amounts it may cause drying trouble, 
because of its high shrinkage and reluctance to dry at all.
%-------------------------------------------------------------------------------
\subsection{Feldspar}
Feldspar is a very common mineral found in most primary rocks in amounts up to 
60\%. It is used as a flux in clay and glazes. Many different types of 
feldspars exist, the main groups are: potash spar, soda spar and lime spar. 

Melting point: 1100--1260\degree C. A mixture of 65\% potash spar and 
35\% soda spar has the lowest melting point.

Feldspar is mainly used in porcelain bodies and white firing stoneware but the 
cost of grinding the feldspar limits its use for other type of ceramics. As a 
substitute, most rocks (both granitic and volcanic) can be used. A cheap source 
of ground rock is the dust produced where stones are crushed for road or 
building construction.

Feldspar can be recognized by its pearly lustre and opaque appearance. The 
crystal has two main cleavages, that are nearly at right angles to each other. 
The colour is whitish, grey or different shades of red. Once a feldspar has 
been shown to you, you will easily recognize feldspars on your own.
%-------------------------------------------------------------------------------
\subsection{Nepheline Syenite}
Nepheline syenite resembles feldspar but contains less quartz; so its melting 
point is lower, and it can be used as a body flux at temperatures above 
1060\degree C. Rocks containing nepheline occur widely. Nepheline rocks contain 
about 2\% iron oxide, which will color white bodies. Commercial nepheline 
syenite has had its iron oxide removed by magnetic separation.
%-------------------------------------------------------------------------------
\subsection{Glass Powder}
Ground glass is a cheap source of flux for both clay bodies and glazes. Broken 
glass can be collected from breweries or window glass shops. The low melting 
point of glass makes it a good body flux for production of vitrified 
earthenware, but it has to be ground to a fine powder and mixed very well with 
the other materials.
%-------------------------------------------------------------------------------
\subsection{Limestone}
Calcium carbonate is the chemical name for whiting, limestone, chalk, marble 
and coral. All these materials can substitute for one another in clay bodies. 
They are cheap and easy to grind, and can be used in both clay bodies and 
glazes as flux. Carbon dioxide is given off before 900\degree C., and what is 
left is calcium oxide. The weight loss amounts to 44\%. The carbon dioxide gas 
may produce pinholes in once fired raw glazed ware. Wollastonite 
(\ce{CaO*SiO2}) does not give off carbon dioxide during firing and can 
substitute for limestone in fast firing bodies.

Calcium oxide should be kept under 2\% in stoneware bodies. It is mainly used 
in earthenware. Its fluxing action is strong, and gives the clay a narrow 
firing range. Above 1100\degree C. fluxing action increases rapidly, causing 
high limestone ware to collapse suddenly. Calcium oxide in the body decreases 
the tendency of glaze crazing, and it decreases the red coloring effect of iron 
oxide.
%-------------------------------------------------------------------------------
\subsection{Talc}
Talc is usually cheap, and it occurs in many locations throughout the world. 
Solid forms of talc are known as steatite or soapstone. Talc deposits may 
contain impurities like limestone, quartz, clay and iron. Its colour varies 
from white, greenish, or grey to brown, but it fires to a cream or white colour 
or grey if its iron content is high. Talc from different deposits varies, 
therefore testing is needed before using talc from a new source.

Talc improves resistance to thermal shock and acid. It reduces the tendency of 
glaze crazing, by preventing expansion of the clay after firing due to moisture 
absorption. During firing the water in the talc mineral is released, causing a 
weight loss (loss on ignition) of about 6\%. Talc starts to work as a flux 
around 1030\degree C, and it produces a stiff glass compared to other fluxes. 
That gives it a long vitrification range, and the ware will not suddenly 
collapse when overfired as is the case of limestone. Talc is able to produce a 
vitrified body and at the same time reduce the tendency of warping and firing 
shrinkage. Small additions of talc reduce the melting point of the clay body; 
large additions make it more refractory.

Talc powder has a small amount of plasticity. In plastic forming and slip 
casting talc causes no problems, but in semi-dry press molding, some types of 
very fine grained talc may produce lamination problems. This can be overcome by 
ensuring proper wetting of the talc particles, followed by granulation of the 
body allowing air to escape during pressing.

Talc is especially valuable for fast firing bodies and for production of wall 
tiles and electrical insulators.
%-------------------------------------------------------------------------------
\subsection{Dolomite}
Dolomite behaves in a body as a more or less equal mixture of limestone and 
talc would do, except that no silica is introduced. Loss on ignition is around 
45\%.
%-------------------------------------------------------------------------------
\subsection{Quartz}
Quartz is a crystal form of silica or silicon dioxide. Silica is found as part 
of rocks and clays and it is so common that it makes up 60\% of all materials 
in the crust of the Earth. As a free mineral, not combined in clay and other 
materials, it occurs as quartz, silica sand, sandstone, and flint pebbles. The 
cheapest source of silica is sand. All sands contain silica in the form of 
small quartz crystals, but a particular sand may contain small crystals of 
other minerals e.g. mica. That should not cause problems in a clay body, unless 
the potter attempts to produce white flawless porcelain.

Additions of silica make the clay more refractory and open up the body, thereby 
reducing shrinkage and drying problems. As is the rule with all other ceramic 
materials, the finer the particles, the more actively the silica will combine 
with other minerals and form a glassy substance with the fluxes in the clay 
during firing. When the silica in the body remains free (uncombined 
chemically), it expands and shrinks suddenly at certain temperatures, as shown 
on the graph in fig~\ref{fig}. These sudden changes may cause the fired body to 
crack, but they also assist in prevention of crazing as explained below.
%-------------------------------------------------------------------------------
\subsection{Grog}
Grog is crushed, already fired clay. The quality and behavior of the grog 
depends on the original clay. It is extensively used in production of 
firebricks, saggars and other refractory products, for reducing firing 
shrinkage and increasing thermal shock resistance. In clay bodies for crockery 
and tiles it is mainly used to improve the forming and drying characteristics, 
without changing the final composition of the body. Grog gives "bone" to the 
body during plastic forming, eases its shaping and prevents it from collapsing 
during throwing on the wheel. It reduces the problems of warping and cracking 
during drying (as sand also does), but without adding the problems of cooling 
cracks that quartz sand may cause.

Grog is normally produced by crushing unglazed waste products in a hammer mill 
or in a pan grinder. The grog particles should be sharp edged.
%-------------------------------------------------------------------------------
\subsection{Coloring Oxides}
Oxides used for coloring clay bodies have to be fairly cheap. That excludes 
most oxides, leaving us with iron oxide, manganese dioxide and ilmenite. 
Coloring of engobes for decoration will not be dealt with here.

Iron oxide exists in two main forms. Red iron oxide (\ce{Fe2O3}) is the same as 
rust, and has a dark red color. Black iron oxide (\ce{Fe3O4}) has a coarser 
particle size than red iron oxide. Black iron oxide can be produced by roasting 
iron metal to 400\degree to 700\degree C in the flue channel or chimney of the 
kiln. The black crust of oxide is knocked off the metal and ball milled. Ochre 
is a yellowish material often used for painting houses. It contains iron oxide 
in a mixture of clay, sand and sometimes limestone. When the ochre contains 
manganese, it is called umber. Both materials can be used as coloring agents in 
clay and glazes. Ilmenite (\ce{FeO*TiO2}) is a black crystal, often occurring 
as black stripes in beach sand together with zircon sand.

The coloring effect of iron oxide depends very much on the atmosphere and 
temperature in the kiln:
%-------------------------------------------------------------------------------
\begin{itemize}
\item Iron oxide in an oxidizing firing below 1020\degree C will produce a 
brick red colour.
\item Oxidizing firing to 1100\degree C turns the red colour darker and 
brownish.
\item In a reducing firing the colour will be grey or black.
\end{itemize}
%-------------------------------------------------------------------------------
Above 1000\degree C iron oxide acts as a strong flux in reducing atmosphere, 
but when the condition is oxidizing, its fluxing action only starts above 
1200\degree C. Whiting present in the clay has a bleaching effect on the red 
colour of iron oxide. In red firing surface clays iron oxide content is often 
10\%.

Manganese dioxide (\ce{MnO2}) is dark brown to black. As a colorant in clay it 
produces yellow, brown, purple or black colors. It acts as a strong flux. Only 
half a percent of manganese will give red clay a brown colour, and with 
increasing amounts the clay will become black. Black colors are obtained by 
adding a mixture of iron oxide and manganese dioxide.
%-------------------------------------------------------------------------------
\section{Classification of Ceramics}
The traditional classification of ceramic ware is in three groups--earthenware, 
stoneware and porcelain--is mainly based on the firing temperature. In common 
usage, the terms often overlap e.g. a term like ``low fired stoneware'' is 
sometimes used to describe ware that resembles real stoneware, but is made from 
clay bodies vitrifying at earthenware temperatures (additions of fluxes). Here 
we shall only deal with earthenware and stoneware.
%-------------------------------------------------------------------------------
\subsection{Earthenware}
Earthenware means pottery that is porous when fired. The firing range normally 
is 9000--1100\degree C. A wide field of different products fits in this group. 
Traditional pottery made from local, red firing clay ware (whether glazed or 
unglazed) is the most common type of earthenware. Unglazed red pottery is often 
called terracotta. Other types, glazed with white opaque glazes, are known as 
Faience or Majolica. Glazed wall tiles are normally made from a porous earthen- 
ware body.
%-------------------------------------------------------------------------------
\subsubsection{Fuel saving}
Earthenware can be as durable as stoneware, and the lower firing temperature 
saves fuel. The cost of additional fluxes needed for maturing the glaze at 
earthenware temperature compared to that of stoneware is easily paid by the 
lower firing cost.
%-------------------------------------------------------------------------------
\subsubsection{Dark bodies}
Yellow, red, brown or black bodies are made from local plastic clay containing 
iron oxide. The starting point is a fairly plastic clay from a nearby source. 

First test the clay, and then modify it as necessary:
%-------------------------------------------------------------------------------
\begin{itemize}
\item if it is too plastic, add sand, grog or a less plastic clay.
\item if it has little plasticity, add a plastic clay or remove some of its 
sand content by washing and screening.
\item if it warps during firing, add a more refractory clay or sand.
\item if it has little firing strength, add a vitrifier like limestone, talc or 
a more fusible clay.
\end{itemize}
%-------------------------------------------------------------------------------
When looking for clay materials, keep costs in mind. Instead of adding feldspar 
as a vitrifier it is cheaper to add a more fusible clay. Instead of adding sand 
or grog, a less plastic clay could be added.
%-------------------------------------------------------------------------------
\subsubsection{White earthenware}
A body with the appearance of porcelain or stoneware can be produced for 
earthenware temperatures. Such a body is made with kaolin, a white firing ball 
clay, and large amounts of talc. Additional flux can be feldspar (preferably 
nepheline syenite), limestone, dolomite, frit or glass powder.

These recipes can be used as a starting point for experiments. For temperatures 
in the 1100\degree to 1200\degree C. firing range the flux content should 
be reduced.

High lime content (up to 20\%) reduces the coloring effect of iron oxide.
%-------------------------------------------------------------------------------
\subsubsection{Thermal crazing}
Crazing of the glaze is a major problem with earthenware. Another problem is 
that glazed rims of pots easily chip off. Both problems are caused by different 
rates of thermal expansion of the glaze and the body. When a pot is taken out 
of the kiln after firing it is exposed to a sudden temperature drop. The glaze 
layer and the body will contract, but most often at different rates. 

Shown in fig~\ref{fig} is what happens when glaze contracts more than body, and 
body contracts more than glaze.
%-------------------------------------------------------------------------------
\begin{itemize}
\item Glaze contracts more

Fig~\ref{fig} shows a body (white) with a glaze on top (black). The glaze and 
the body has contracted at the same rate and there is no tension between the 
two.
 
In this case, the glaze contracted more than the body (leaving it shorter than 
the body), which puts the glaze with a tensile stress (it is pulled apart). If 
the body is very thin it will bend as shown.

More likely, the tensile stress will be relieved by cracks in the glaze, as 
shown in fig~\ref{fig}. This is called crazing. The stress, caused by high 
expansion (and contraction) of the glaze, may be relieved by crazing as soon as 
the pot is taken out of the kiln or it may take days, months or years. If it 
takes a long time for crazing to appear, this means that the expansion of clay 
and glaze is almost equal.

When the body has been exposed to humidity for a long period, water enters the 
body, which expands slightly (moisture swelling). This expansion may cause a 
glaze to become too short and it will craze. This kind of crazing is called 
delayed crazing or moisture crazing.

\item Body contracts more

This example shows a body with higher expansion rate than the glaze. The body 
contracted more than the glaze when it cooled. The glaze is under compression, 
and if the clay is thin it may bend as shown to relieve the pressure. If body 
contraction is only slightly higher than glaze contraction, nothing will happen 
except the glaze will not craze. If a glaze contracts much less than the body, 
the compression on the glaze becomes too high and the glaze will start to chip 
off like this. This will not happen by itself, but only if something bangs 
against the pot. Typically, the rim of a cup will chip off. <here insert 57-6> 
In extreme cases, high compression of the glaze may cause the body to crack.
\end{itemize}
%-------------------------------------------------------------------------------
\subsubsection{Crazing cure}
Crazing is cured by adjusting the expansion rate of body and glaze. If the 
glaze is too big for the body, it is under a constant squeeze and will be less 
likely to craze. This squeeze is obtained if cristobalite is present in the 
body.
%-------------------------------------------------------------------------------
\subsubsection{Cristobalite}
Cristobalite is a crystal form of silica (\ce{SiO2}) which is formed above 
870\degree C, 
when some of the free silica in the body changes its crystal form.

The graph in fig~\ref{fig} shows the contraction of glaze and body during 
cooling. Both contract gradually until 573\degree C., when quartz crystals in 
the body suddenly contract about 1\%. The glaze is still soft enough to 
accommodate this contraction. The glaze hardens around 500\degree C, and from 
then on it contracts at its own rate. In this case, the glaze contracts more 
than the body and it will craze.

The graph in fig~\ref{fig} shows contraction of a glazed body containing 
cristobalite. Initially the graph is similar to the upper one, but at 
226\degree C. When the glaze is hard, the cristobalite in the body contracts 
3\%, leaving the glaze in compression. In earthenware, the effect is usually 
only present if the body is close to vitrification.

The conversion of quartz into cristobalite is helped by fine grinding of 
quartz, by a higher firing temperature, and by presence of talc or limestone. 
Flint, a form of quartz, converts more easily into cristobalite in earthenware 
bodies. If the glaze is under too much of a squeeze it may crack the pot or 
cause shivering and peeling of the glaze.
%-------------------------------------------------------------------------------
\subsubsection{Moisture crazing}
After firing, the porous earthenware body will absorb moisture and this causes 
the body to expand. If the glaze is not under sufficient compression it will 
craze. Such delayed crazing may occur a long time after firing. The moisture 
expansion of the body is reduced by making the body more vitreous. Additions of 
talc or limestone to the body reduce moisture crazing.
%-------------------------------------------------------------------------------
\subsubsection{Crazing cure}
For both types of crazing the cure is:
%-------------------------------------------------------------------------------
\begin{itemize}
\item add quartz (or silica), talc, limestone to the body.
\item biscuit fire to a higher temperature.
\item glaze fire to a higher temperature.
\item add silica to the glaze.
\item replace alkaline fluxes (soda and potash) in the glaze with boron oxide.
\end{itemize}
%-------------------------------------------------------------------------------
\subsubsection{Chipping cure}
When the glaze is peeling or chipping off:
%-------------------------------------------------------------------------------
\begin{itemize}
\item reduce quartz content of the body.
\item reduce silica content of the glaze.
\item increase alkaline fluxes in the glaze.
\item add feldspar to the body (above 1100\degree C)
\end{itemize}
%-------------------------------------------------------------------------------
\subsection{Stoneware}
If fuel cost is not important and good refractories for kiln furniture are 
available, the firing range from 1180\degree to 1300\degree C offers several 
advantages over earthenware. At this temperature less fluxes are needed, and it 
is often possible to find a natural clay that fires to a strong dense body with 
little addition of other materials. Crazing of the glaze is less of a problem, 
since at this temperature there is a better body to glaze bond. Even if crazing 
occurs, the vitrified body will remain waterproof. Another advantage is that 
the higher firing temperature makes it possible to rely on non-fritted fluxes, 
like feldspar and limestone, for the glaze.

Examples of stoneware are hotel crockery, floor tiles, sanitary wares, 
salt-glazed sewage pipes and utensils, electrical insulators and corrosion free 
vessels for the chemical industry.
%-------------------------------------------------------------------------------
\subsubsection{Vitrification}
Stoneware is hard, strong and dense. Its color varies from buff or grey to 
brown or black. Due to the high firing temperature, the fluxes in the body 
gradually melt and fill the space between the clay particles with a glassy 
mass. This strongly binds the clay particles together, but if the body is 
overfired it becomes brittle like glass and loses its strength. Feldspar is the 
preferred flux for stoneware, because it has a long vitrification range and it 
produces a viscous glass, that does not cause the ware to collapse suddenly. To 
vitrify means to become glass-like.
%-------------------------------------------------------------------------------
\subsubsection{Native clays}
Many native clays can be used for stoneware, as dug or in combination with 
other clays. Stoneware bodies soften during firing, and large items tend to 
warp or sag unless grog or sand is added to the clay. Earthenware clays low in 
iron oxide can often be used with additions of fireclay or kaolin.

The optimum recipe for a stoneware body is found by making tests of clay bodies 
with varying additions of fluxes and/or refractory clay and openers like sand 
and grog (as listed for earthenware).
%-------------------------------------------------------------------------------
\subsubsection{Crazing}
Crazing is seldom a problem for stoneware since the body itself is waterproof. 
Furthermore, the body and glaze materials partly melt together and form a 
strong bond. At stoneware temperature, quartz readily changes into 
cristobalite, which further reduces the tendency to craze. When crazing does 
occur the corrections listed under earthenware apply.
%-------------------------------------------------------------------------------
\subsubsection{Cracking}
A more common problem with stoneware is cracking during firing, or more often 
during cooling. 

The main cause is the change in silica crystals at 226\degree C 
and 573\degree C. Free silica in the form of cristobalite expands nearly 3\% 
when heated to 226\degree C and contracts again when cooled to the same 
temperature. Free silica in the form of quartz goes through a 1\% expansion and 
contraction at 573\degree C (see fig~\ref{fig}). 

These changes take place suddenly, and because large items do not cool evenly, 
one part of a pot may reach contraction temperature before another. When one 
area contracts suddenly, the stress within the pot may cause it to crack. If 
pots only crack occasionally, the cure is to cool the kiln more slowly, 
particularly from 700\degree--150\degree C. Otherwise changes have to be made 
in 
the body.
%-------------------------------------------------------------------------------
\subsubsection{Cracking remedies}
Since the problem is caused by free silica, the cure is to reduce the amount in 
the body. Free silica comes from quartz or flint added to the body or 
present in the raw clay, or the release of silica from other clay minerals 
during firing. One or more of the following changes can be made:
%-------------------------------------------------------------------------------
\begin{itemize}
\item Grog can replace quartz sand.
\item Clay containing high amounts of sand could be washed or replaced by a 
less sandy clay.
\item The release of free silica from the clay minerals cannot be avoided, but 
since montmorillonite clays (bentonite) release twice as much free silica as 
kaolin, it may help to reduce the amount of montmorillonite clay in the body. A 
well equipped ceramics laboratory can determine the content of montmorillonite 
clay minerals in a clay, but potters without access to such services will have 
to rely on practical testing of the different clays available. A clay which is 
difficult to dry, has high plasticity and a greasy feeling to it, should be 
suspected of containing montmorillonite.
\item Free silica in the body is also reduced by adding a flux that will combine
with the silica and form a glass. When silica is part of a glass it will not 
cause cracking, because glass is non-crystalline.
\end{itemize}
%-------------------------------------------------------------------------------