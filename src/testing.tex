\chapter{Testing of Clay}
The tests which are described in this chapter will not all be appropriate for 
small pottery industries. Cottage industries will be content with making, at 
most, shrinkage and moisture content tests. However, industries or ceramics 
training centres involved with development and research work should find most 
of the tests useful and relatively simple to undertake.

\textbf{Caution:} Testing should only be done for good reasons. Most private 
cottage and small industries do not have time or resources to carry out regular 
testing. However, many production losses can be prevented by testing materials 
when they seem to behave differently than usual. When clay arrives it is 
inspected, and by making a simple rope test its plasticity can be judged. Am 
experienced thrower will immediately feel differences in the clay body, and 
should suggest testing to the manager.

In carrying out these tests the most important thing to keep in mind is; 
consistency. That means, the tests should be made in the same manner each time, 
so that the results can be compared.
%-------------------------------------------------------------------------------
\section{Testing}
The individual tester builds up his own set of standards by which he evaluates 
his test results.
%-------------------------------------------------------------------------------
\subsection{Equipment}
For most of the tests a scale, a ruler and a ceramic kiln are sufficient. A 
small drying oven is useful for moisture tests and thermal crazing tests. A 
transverse strength tester is needed for testing the strength of the green or 
fired clay bodies. These two pieces of equipment can be made locally.
%-------------------------------------------------------------------------------
\subsection{Goal}
The object of testing natural clays and clay bodies is:
%-------------------------------------------------------------------------------
\begin{itemize}
\item to find out the properties of clay from new deposits and to assess their 
usefulness in production. (research)
\item to be sure that new supplies of familiar clays are similar to former 
supplies of the same clay. (quality control)
\item to test the quality of new clay body mixtures (research)
\item to control the quality of clay bodies used in production (quality 
control).
\end{itemize}
%-------------------------------------------------------------------------------
The tests required for the various situations differ. The table below suggests 
what test to make.

The quality of clay taken from locations only a few feet apart in the clay 
deposit may differ. Each lorry load of clay delivered may also differ in 
quality.

Therefore, the clay samples for testing should be collected from at least 4 
different locations within the clay deposit or from where the clay has been 
dumped.
%-------------------------------------------------------------------------------
\subsection{Quartering}
Quartering ensures that the remaining portion is ``average quality''. It is 
then used for carrying out the clay tests.
%-------------------------------------------------------------------------------
\begin{enumerate}
\item Mix the four samples well .
\item Divide the mixture by a cross cut in the pile.
\item Two portions facing each other are removed and the remaining two portions 
are mixed well.
\item This is again divided by a cross cut and steps 1-4 are repeated another 3 
times.
\end{enumerate}
%-------------------------------------------------------------------------------
\section{Moisture Content}
Even if the clay looks and feels dry, it will contain some water. When clay is 
purchased it may contain any amount of water, and if the clay is bought by 
weight the water is also paid for. So knowing the water content helps your 
profits.

Ideally, each batch of clay delivered to the pottery is tested. This test is 
also done for checking moisture content of plastic bodies, and semi-dry clay 
bodies for press molding.
%-------------------------------------------------------------------------------
\begin{enumerate}
\item Sample clay by quartering.
\item Weigh out 200 g ($W moist$) of the moist clay.
\item Place the clay in a clean cup and heat it to 150\degree C for 2 hours.
\item Find the weight ($W dry$) in g of the dry clay.
\end{enumerate}
%-------------------------------------------------------------------------------
\section{Particle Size}
A quick test to check clay and sand content of new clay supplies is done by 
screening procedure.
%-------------------------------------------------------------------------------
\begin{enumerate}
\item About 500 g clay is dried at 150\degree C for 2 hours.
\item Weigh the clay.
\item The clay is made into a thin slurry
\item The slurry is screened through one or more fine sieves.
\item The residue left on each screen is dried and put on a set of scales.
\end{enumerate}
%-------------------------------------------------------------------------------
This figure indicates the amount of sand in the clay. A 200 mesh sieve holds 
back particles bigger than 0.0076 mm and some fine sand will pass this screen, 
but for comparing the quality of new batches of clay with former supplies it is 
accurate enough.
%-------------------------------------------------------------------------------
\section{Plasticity}
\subsection{Atterberg Number}
The more plastic a clay, the more water it will absorb without becoming fluid. 
So the range of water content over which a clay is plastic is a measure of its 
plasticity.

The water range, $Lw - Pw$ is called the Atterberg number. The water contents 
of $Lw$ and $Pw$ are expressed as a percentage of the weight of the wet clay.

The higher the number, the more plastic the clay. Kaolin has an Atterberg 
number of 10--15, and ball clay has an Atterberg number of 10--15.
%-------------------------------------------------------------------------------
\begin{enumerate}
\item A portion of clay, about 300 g, is stirred with water to form a liquid 
slip of creamy consistency. The slip is poured into a clean plaster mould. 
\item After a minute or so a knife is dipped into the slip at intervals, noting 
whether the incision disappears again. When the incision just remains, a sample 
of the clay around the incision is removed and its weight, $Wwet$ is found.
\item The remaining clay in the plaster mould is turned around in the mold 
until it can be formed in the palm of a hand. Knead the clay between the palms 
of your hand until it starts to crumble. 
\item At that point the plastic limit of the clay is reached and the sample is 
weighed to find W plastic. 
\item Dry the clay in the drying oven at 150 C for two hours and find its 
weight, $Wdry$.
\end{enumerate}
%-------------------------------------------------------------------------------
\subsection{Plastic Limit}
The plastic limit, $Pw$, is the minimum amount of water required to make the 
clay plastic so that it can be formed.
%-------------------------------------------------------------------------------
\subsection{Liquid Limit}
The liquid limit $Lw$, is the minimum amount of water required to make a clay 
slip flow under its own weight.
%-------------------------------------------------------------------------------
\section{Loss on Ignition}
Pore water is lost around 100\degree C. Chemically bonded water is released at 
350\degree --600\degree C from inside the crystal structure of the clay 
minerals. Loss of chemically bonded water corresponds directly to the amount of 
clay minerals present in the clay. Chemical water is not released from sand or 
feldspar.

Kaolinite: \ce{Al2O3*2SiO2*2H2O -> Al2O3 + 2SiO2 + 2H2O} has a loss in form of 
water of 14\%.

Montmorillonite: \ce{Al2O3*4SiO2*H2O -> Al2O3 + 4SiO2 + H2O} has a loss in form 
of water of 5\%.

A sandy clay with kaolinite clay crystals and showing a loss of 7\% would, 
therefore, contain only 50\% clay minerals the rest being non plastic materials 
like sand, mica or feldspar.

Limestone: \ce{CaCO3 -> CaO + CO2} has a loss in form of CO2 of 44\%.

If the limestone test sample shows less than 44\% loss on ignition we can 
estimate the amount of sand or other impurities. Lime decomposes at 825\degree 
--900\degree C.

The loss on ignition test is mainly used to check raw materials supplies and 
their quality from batch to batch. If the loss is lower than the standard, it 
shows that the material of that batch contains more sand or other impurities 
than normal.
%-------------------------------------------------------------------------------
\begin{enumerate}
  \item Dry a sample, about 200g, in the oven at 150\degree C for 2 hrs 
in a clean bowl.
\item Find its weight, $Wdry$. 
\item Fire the test sample to 1000\degree C or higher in an unglazed bowl. 
\item After cooling but when it is still warm from the kiln, find the weight, 
$Wloss$. 
\item Compare with results of former tests.
\end{enumerate}
%-------------------------------------------------------------------------------
\section{Shrinkage}
Drying shrinkage depends on the fineness and plasticity of the raw clay. Firing 
shrinkage indicates the degree of vitrification of the clay or clay body at the 
temperature at which it has been fired (the higher the temperature, the more 
shrinkage).
%-------------------------------------------------------------------------------
\begin{enumerate}
\item Mix the clay with water to plastic consistency and knead it well.
\item Form 10 test bars of each clay to be tested. The test bars are molded in 
a standard mould.
\item Mark the test bars with two incised lines exactly 100 mm apart.
\item Turn the test bars several times while they dry to prevent warping.
\item After the test bars are completely dry (open air) measure the distance 
between the two cuts in mm.

Drying shrinkage in percent = $100 - dry~length~(mm)$

For the most accuracy, take the average of 10 test bars.
\item Firing shrinkage is found by firing 5 of the test bars to the intended 
firing temperature.
\item Measure the distance between the lines on the test bars in mm.
\end{enumerate}
%-------------------------------------------------------------------------------
Here it is recommended to make many test bars in order to ensure more reliable 
results.
%-------------------------------------------------------------------------------
\section{Softening Point}
Clay becomes soft at high temperatures, because various impurities like 
feldspar, lime and iron oxide start to form a liquid mass between the clay 
particles. This will cause the ceramic ware to warp during firing and if the 
clay is used for refractories (like saggars), these may bend under load.

This test is used to compare the refractoriness of different clays.
%-------------------------------------------------------------------------------
\begin{enumerate}
\item Two test bars, formed as described in the shrinkage test (could be two 
out of the five used for shrinkage test), are suspended on two points placed 
100 mm apart. This leaves the test bar between the two marks unsupported.
  
\item During firing, the test bar will bend and the degree of its bend is a 
crude measure for the clay's tendency to sag under load.
\end{enumerate}
%-------------------------------------------------------------------------------
The degree of its bend (see fig~\ref{fig}) is measured in mm.
%-------------------------------------------------------------------------------
\section{Porosity}
A piece of clay fired to about 600\degree C will be very porous. As the 
temperature increases, the feldspar, lime and other impurities will begin to 
melt together with the silica in the clay, forming a glass. This will gradually 
fill the pores of the clay, making it less porous. This process is called 
vitrification, and the more a clay vitrifies the less porous it becomes.

Porosity can be tested by measuring the fired clay's ability to absorb water 
procedure:
%-------------------------------------------------------------------------------
\begin{enumerate}
\item Heat a sample of a fired body without glaze to 150\degree C for 1 hour, 
or take 
it directly from a warm kiln after firing.

After drying, find its weight, $Wdry$.

\item Immerse the sample completely in water, and leave it for 24 hrs.

\item After 24 hours, take the test piece out of the water, and dry its 
surfaces with a piece of cloth.

\item Find the weight, $Wsoak$, of the soaked test piece.
\end{enumerate}
%-------------------------------------------------------------------------------
Examples of water absorption are shown in ref~\ref{tab:waterabsorption}
%------------------------------------------------------------------------------
\begin{center}
  \begin{table}\centering
    \renewcommand{\arraystretch}{1.5}    
    \begin{tabular}{|c|c|}\hline
      \textbf{Material}&\textbf{Absorption}\\\hline\hline
Red brick&20--40\%\\\hline
Earthenware&5--20\%\\\hline
Stoneware&1--5\%\\\hline
Wall tiles&15--20\%\\\hline
Floor tiles&3--5\%\\\hline
    \end{tabular}
    \caption{Levels of water absorption for various materials.}
    \label{tab:waterabsorption}
  \end{table}
\end{center}
%------------------------------------------------------------------------------
A body with water absorption of 1\% or less, and with low lime content, is 
termed acid proof. However, floor tiles for dairies and most other uses, where 
acid proof materials are required, can safely have a water absorption up to 3% .

Bodies for kiln furniture should have water absorption of 18\% or more, 
otherwise they are likely to break due to thermal shock.
%-------------------------------------------------------------------------------
\section{Transverse Strength}
A transverse strength tester is used for testing the strength of dry green test 
bars of clay bodies or raw clays, and fired bodies.
%-------------------------------------------------------------------------------
\subsection{Green Strength}
The strength of green bodies is also an indirect measure of the plasticity of 
the body. The more fine clay particles a clay contains, the greater its 
strength. Clay from new deposits can be compared with known clays. By comparing 
the strength of new clay supplies with old ones from the same source,we can 
control the quality of our supplies.

Kaolin has a green transverse strength of 7--15 kg/cm$2$, and ball clay has a 
green transverse strength of 20--90 kg/cm$2$
%-------------------------------------------------------------------------------
\subsection{Fired Strength}
By testing the fired clay body, we can check its strength. The strength of 
fired bodies (either glazed or unglazed) depends to a high degree on the 
firing temperature. Around 900\degree C feldspar, lime and other impurities in 
the clay start to melt, forming a glassy mixture which "glues" the clay 
particles together. This process is called vitrification.

The graphs in fig~\ref{fig} show the relationship between fired strength, 
firing temperature, and vitrification. As the temperature rises, more and more 
glass will form until all spaces between the clay particles are filled with it 
(total vitrification). At that point the clay becomes similar to glass, and it 
is brittle with little strength.
%-------------------------------------------------------------------------------
\subsection{Transverse Strength Tester}
Test bars similar to the ones used for shrink- age tests are used for testing 
of transverse strength. The test bar is supported at two points (adjustable) 
and a load is applied to a point half-way between the supports (fig~\ref{fig}). 
The load is a bucket which is gradually filled with water (or dry sand). When 
the test bar breaks, the filling of water is stopped and the weight of the 
water is found.

Testing of fired bodies should be done close to the pivot ($short~L2$) whereas 
testing green bodies can be done farther from the pivot ($long~L2$). The test 
bar is placed evenly on the supports and the load $W2$ is placed exactly 
half-way between the supports. An empty bucket is placed on the hook, and is 
counterbalanced with the counter weight. The load is made to just touch the 
test bar by adjusting the balance with a screw-weight.

After reading L1 and L2 on the ruler and measuring width (b) and thickness (t) 
of the test bar, the transverse strength can be calculated (all measures in cm 
and kg).

The same test bar can be tested 3 times by first testing it with (a) at maximum 
and then test the two resulting pieces again. Normally at least 3 test bars 
should be tested and the final result is recorded as an average of all 9 
results.

The transverse strength is also called modulus of rupture.
%-------------------------------------------------------------------------------
\subsection{Moisture}
The moisture content of the test bar greatly influences the transverse 
strength. Therefore, the test bars should be dried to the same degree of 
moisture. With changing weather, that may be difficult to ensure. A safe way of 
establishing consistent results is, as a standard, to dry all test bars at 
100\degree C for 2 hours.
%-------------------------------------------------------------------------------